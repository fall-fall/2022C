\section{灵敏度分析}

前述模型是基于一系列经济与生产参数构建的,这些参数在现实世界中天然存在波动性。为了评估模型所得方案的可靠性与稳定性,本章将进行系统的灵敏度分析。此分析旨在探究最优种植方案对关键参数变化的响应程度,检验方案在不同市场环境下的表现,并验证模型结构中关键假设的合理性与价值。

\subsection{问题一确定性模型的灵敏度分析}

问题一的混合整数线性规划模型建立在所有输入参数均为确定值的基础上。然而,市场价格、生产成本与作物产量等均是动态变化的。因此,本节的分析旨在检验确定性最优种植方案对这些核心经济参数波动的敏感程度,识别对总利润影响最显著的因素,并评估方案在宏观市场变化情景下的稳健性。

分析的第一步是单因素扰动分析,用以探究单一参数变化对整体结果的独立影响。首先,基于基准模型求解得到的最优种植方案,识别出对七年总利润贡献度最高的若干种核心作物。这些作物的销售价格($P_{jy}$)、种植成本($C_{jy}$)与单位面积产量($\text{yield}_{jy}$)被选为关键扰动参数。在分析过程中,每次仅选择一个参数,在其基准值的基础上进行特定范围的扰动,例如$\pm 5\%$至$\pm 20\%$。在保持其他参数不变的条件下,重新求解优化模型,记录总利润的变化。为量化模型对参数的敏感程度,定义灵敏度系数$S$如下:

\begin{equation}
    S = \frac{\Delta \text{Profit} / \text{Profit}_{\text{base}}}{\Delta \text{Parameter} / \text{Parameter}_{\text{base}}}
\end{equation}

其中,$\Delta \text{Profit}$与$\Delta \text{Parameter}$分别代表总利润与参数值相对于基准值的变化量。该过程对所有已识别的关键参数重复进行。

% \begin{figure}[H]
%     \centering
%     \includegraphics[width=0.8\textwidth]{figs/灵敏度/单因素分析图.png}
%     \caption{关键参数的单因素灵敏度分析结果}
%     \label{fig:sen_p1_single}
% \end{figure}

图\ref{fig:sen_p1_single}展示了各关键参数的灵敏度系数。系数的绝对值$|S|$越大,表明总利润对该参数的波动越敏感。从图中结果可知,部分高价值蔬菜的销售价格与单位面积产量是影响最终收益的关键因素。这些被识别出的高敏感性参数构成了种植决策中的核心风险源,在实际生产管理中需要得到重点关注与监控。

为进一步模拟真实市场环境的复杂性,我们进行了多因素情景分析。该分析旨在评估基准最优方案在面对系统性冲击时的整体表现。为此,构建了三个具有代表性的宏观市场情景:情景A(通货膨胀),假设所有作物的种植成本普遍上涨15\%;情景B(市场利好),假设蔬菜类作物的销售价格普遍上涨20\%,粮食类价格稳定;情景C(供给冲击),假设因气候不利,所有露天作物的单位面积产量下降10\%。将问题一求解出的固定最优种植方案$a_{ijky}^*$分别置于这三种情景下,重新计算其七年总利润,而不进行重新优化。

% \begin{figure}[H]
%     \centering
%     \includegraphics[width=0.8\textwidth]{figs/灵敏度/情景分析图.png}
%     \caption{最优方案在不同市场情景下的利润表现}
%     \label{fig:sen_p1_scenario}
% \end{figure}

图\ref{fig:sen_p1_scenario}对比了基准利润与各情景下的利润。结果显示,在成本上涨(情景A)和产量下降(情景C)的不利条件下,方案的总利润出现了不同程度的下滑。利润下降的幅度是衡量该方案稳健性的直接指标。若下降幅度相对较小,则表明该方案能够较好地抵御相应类型的市场风险。反之,较大的利润损失则说明该方案高度依赖于初始的市场假设,在面对特定类型的风险时表现脆弱,提示决策者需要为这类市场风险准备应对预案。

\subsection{问题二鲁棒优化模型的灵敏度分析}

问题二鲁棒优化模型的核心在于不确定集$U$的定义,其目标是在该集合内的所有可能场景中寻求最坏情况下的最优解。因此,本节的灵敏度分析旨在检验最优保底利润及对应的种植方案对不确定集定义的依赖程度,并量化风险规避水平与保证收益之间的权衡关系。

分析的核心是扰动不确定集的大小。模型中的不确定参数(如价格、产量)的波动范围由一个波动系数$\delta$决定。我们将$\delta$作为关键扰动参数,通过设定一系列递增的$\delta$值(例如从5\%至25\%),构建不同大小的不确定集$U$,并对每个$U$重新求解相应的鲁棒优化模型。此过程记录了在每个$\delta$值下,模型所能实现的最优年度平均保底利润$\Gamma^*$。

% \begin{figure}[H]
%     \centering
%     \includegraphics[width=0.8\textwidth]{figs/灵敏度/鲁棒分析图.png}
%     \caption{保底利润随不确定集波动范围 ($\delta$) 的变化关系}
%     \label{fig:sen_p2_robust}
% \end{figure}

图\ref{fig:sen_p2_robust}绘制了最优保底利润$\Gamma^*$与波动系数$\delta$之间的关系。图中清晰地呈现出一条向下倾斜的曲线,这直观地揭示了鲁棒性的“代价”。为了抵御更大范围的市场风险(即更大的$\delta$值),决策者必须接受一个相对较低的保底利润水平。这条有效前沿曲线为决策者在风险与回报之间进行权衡提供了定量的决策依据。

鲁棒模型提供的是一个抗风险方案,但其在随机未来中的期望表现需要通过后验模拟检验。为此,我们选取了不同鲁棒水平下的最优种植方案(例如$\delta=10\%$和$\delta=20\%$得到的方案$a_{10\%}^*$和$a_{20\%}^*$),并以问题一的确定性最优方案$a_{\text{det}}^*$作为基准进行对比。通过蒙特卡洛方法生成大量随机市场情景,将这三个固定的种植方案分别代入每一个随机情景中计算利润,从而得到每个方案的利润概率分布。我们基于期望利润、利润标准差以及条件风险价值(CVaR)等统计指标对方案进行量化评估。

% \begin{figure}[H]
%     \centering
%     \includegraphics[width=0.8\textwidth]{figs/灵敏度/后验模拟图.png}
%     \caption{确定性方案与鲁棒方案在后验模拟中的利润分布与风险指标对比}
%     \label{fig:sen_p2_posterior}
% \end{figure}

图\ref{fig:sen_p2_posterior}展示了后验模拟的结果。结果表明,鲁棒方案的期望利润可能略低于确定性方案,但其利润分布更为集中,表现为更低的标准差和显著改善的CVaR指标。这定量地证明了鲁棒模型在风险管理方面的价值:通过牺牲少量在理想情况下的最优期望收益,换取了在不利情况下风险的显著降低,提升了方案在不确定环境下的整体表现。

\subsection{问题三动态反馈模型的灵敏度分析}

问题三模型引入了描述市场反馈的非线性关系,其核心参数为成本敏感度系数$\alpha_j$与价格敏感度系数$\beta_j$。这些系数是基于对市场行为的假设设定的,其取值的准确性对模型结果有直接影响。本节分析旨在检验模型结果对这些关键行为参数的敏感性,并从根本上验证引入市场反馈这一复杂结构的必要性。

首先,我们对关键的敏感度系数进行扰动分析。这些系数由更底层的假设(例如,某作物产量超出基准10\%时,其价格下降$k_j$\%)推导而来。我们针对利润贡献最高的几种作物,对其价格下降率$k_j$和成本上涨率$m_j$的基准设定进行不同程度的扰动。每次扰动后,重新计算对应的$\alpha_j$和$\beta_j$,并完整运行整个“启发式算法+蒙特卡洛”求解流程,得到一个新的最优期望总利润$E[Z]^*$。

% \begin{figure}[H]
%     \centering
%     \includegraphics[width=0.8\textwidth]{figs/灵敏度/敏感系数扰动图.png}
%     \caption{最优期望利润对关键作物市场敏感度系数的响应}
%     \label{fig:sen_p3_coeffs}
% \end{figure}

图\ref{fig:sen_p3_coeffs}展示了最优期望利润$E[Z]^*$随关键敏感度参数变化的响应情况。若$E[Z]^*$随参数变化表现出剧烈波动,则说明模型结果对市场弹性的假设高度敏感。这一结果提示,为了获得更可靠的决策支持,有必要投入更多资源以获取更准确的市场反应数据,从而降低模型的不确定性。

为了验证模型中引入市场反馈效应的必要性,我们设计了对照情景分析。我们构建了一个不包含市场反馈的简化版模型作为对照组。具体而言,在该模型中,强制将所有作物的成本敏感度系数$\alpha_j$和价格敏感度系数$\beta_j$均设为0,并移除所有与作物替代或互补相关的销量修正规则。该简化模型在本质上退化为一个仅考虑参数随机波动但无市场反馈的随机优化模型。我们使用相同的算法求解该简化模型,得到一个“无关联”情景下的最优期望利润$E[Z]_{\text{no-corr}}^*$及其对应的种植方案$a_{\text{no-corr}}^*$。

% \begin{figure}[H]
%     \centering
%     \includegraphics[width=0.8\textwidth]{figs/灵敏度/对照分析图.png}
%     \caption{完整模型与“无关联”对照模型的最优期望利润对比}
%     \label{fig:sen_p3_control}
% \end{figure}

图\ref{fig:sen_p3_control}对比了完整模型与“无关联”对照模型求解得到的最优期望利润。结果显示,完整模型得到的最优期望利润$E[Z]^*$显著高于对照模型的$E[Z]_{\text{no-corr}}^*$。这一差异有力地证明了在模型中考虑市场关联效应的价值。通过将市场反馈纳入决策过程,模型能够指导种植策略主动规避因过度生产导致的“增产不增收”陷阱,从而找到在真实市场环境下表现更优的方案,创造了额外的经济价值。